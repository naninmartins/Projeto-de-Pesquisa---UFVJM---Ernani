
\documentclass[
	oneside,
	12pt,				% tamanho da fonte
	%%openright,			% capítulos começam em pág ímpar (insere página vazia caso preciso)
	%%twoside,			% para impressão em recto e verso. Oposto a oneside
	a4paper,			% tamanho do papel. 
	english,			% idioma adicional para hifenização
	brazil,				% o último idioma é o principal do documento
	article
	]{abntex2}

% ---
% Pacotes fundamentais 
% ---

\usepackage{times}				% Usa a fonte Time News Roman
\usepackage[T1]{fontenc}		% Selecao de codigos de fonte.
\usepackage[utf8]{inputenc}		% Codificacao do documento (conversão automática dos acentos)
\usepackage{indentfirst}		% Indenta o primeiro parágrafo de cada seção.
\usepackage{microtype} 			% para melhorias de justificação
\usepackage{setspace}

\usepackage{fancyhdr}
\pagestyle{fancyplain}
\fancyhf[]{}
\rfoot{\thepage}



% Pacotes de citações
% ---
\usepackage[brazilian,hyperpageref]{backref}	 % Paginas com as citações na bibl
\usepackage[alf]{abntex2cite}	% Citações padrão ABNT

\instituicao{UNIVERSIDADE FEDERAL DOS VALES DO JEQUITINHONHA E MUCURI - UFVJM
	PROGRAMA DE PÓS-GRADUAÇÃO STRICTO SENSU EM EDUCAÇÃO}
\titulo{Modelo de Ambiente Virtual de Aprendizado suportado por CS}
\local{Brasil}
\data{2019}

\renewcommand{\imprimircapa}{%
	\begin{capa}%
		\center	
		{\ABNTEXchapterfont\imprimirinstituicao}
		\vfill
		\begin{center}
			\ABNTEXchapterfont\bfseries\imprimirtitulo
		\end{center}
		\vspace*{3cm}
		\begin{flushright}
		Linha de pesquisa: EDUCAÇÃO E TECNOLOGIAS APLICADAS EM INSTITUIÇÕES
		EDUCACIONAIS.
		\linebreak
		Possível Orientadora: Profa. Dra. Luciana Pereira de Assis
		\end{flushright}
		\vfill
		\imprimirlocal 
		\linebreak		
		\imprimirdata
		\vspace*{1cm}
	\end{capa}
}

% informações do PDF
\makeatletter
\hypersetup{
     	%pagebackref=true,
		pdftitle={\@title}, 
		pdfauthor={\@author},
    	pdfsubject={\imprimirpreambulo},
	    pdfcreator={LaTeX with abnTeX2},
		pdfkeywords={abnt}{latex}{abntex}{abntex2}{projeto de pesquisa}, 
		colorlinks=true,       		% false: boxed links; true: colored links
    	linkcolor=blue,          	% color of internal links
    	citecolor=blue,        		% color of links to bibliography
    	filecolor=magenta,      		% color of file links
		urlcolor=blue,
		bookmarksdepth=4
}


% O tamanho do parágrafo é dado por:
\setlength{\parindent}{1.3cm}
% Controle do espaçamento entre um parágrafo e outro:
\setlength{\parskip}{0.2cm}  % tente também \onelineskip

% Configurações de margem do projeto
\usepackage[left=2.5cm, right=2.5cm, top=2.5cm, bottom=2.5cm]{geometry}



\begin{document}
 
% Seleciona o idioma do documento (conforme pacotes do babel)
%\selectlanguage{english}
\selectlanguage{brazil}

% Retira espaço extra obsoleto entre as frases.
\frenchspacing 

%% Imprimir capa via abntex
\imprimircapa

\textual

\OnehalfSpacing %%set espaçamento entre linhas de 1cm
\begin{center}
	\ABNTEXchapterfont\bfseries RESUMO
\end{center}

Consciência Situacional (CS) é amplamente utilizada na correta compreensão do ambiente e da situação, sendo considerada principal precursora do processo de Tomada de Decisão, portanto, considerando seu poder de compreensão do universo, a CS tem o potencial de ser aplicada em Ambientes Virtuais de Aprendizagem (AVA’s).
O Ambiente Educacional compõe-se de um espaço extremamente dinâmico, por tanto muitas vezes as inúmeras situações presentes em um estado podem dificultar o processo de tomada de decisão. Atualmente é notório o crescimento de pesquisas que baseiam-se no uso de Mineração de Dados Educacionais (MDE) afim de extrair conhecimento dos dados de AVA's, entretanto somente a MDE por vezes não é suficiente para lidar com a variedade de dados e eventos gerados da interação de cada usuário com o Ambiente Educacional. \citeonline{Martins2018} desenvolve um Modelo baseado em CS e Mineração de Dados Educacionais (MDE) de suporte a aprendizagem em AVA's, o modelo propõe a expansão da MDE através da CS utilizando-se de modelos mentais, indicando fortes benefícios ao ambiente escolar, entretanto o autor não aplica o modelo ainda no ambiente. Este projeto de pesquisa objetiva-se na expansão do modelo, o uso de linguagens lógicas para a construção de regras decisórias e modelos mentais demonstra-se viável para a aplicabilidade do problema, é necessário também a construção e teste do mesmo, assim a elaboração de um protótipo de software para aplicabilidade do modelo são objetivos deste projeto de pesquisa. Espera-se que este estudo permita a aplicabilidade do modelo de CS no âmbito educacional, procurando assim minimizar a sobrecarga de professores e tutores dentro do ambiente educacional, espera-se também observar uma nova perspectiva da MDE, visando atingir um níveis de CS em situações nas quais tais técnicas e procedimentos apresentam uma melhor performance. 
\linebreak\linebreak
\textbf{Palavras-chave:} Consciência da Situação. Ambientes Virtuais de Aprendizagem. Mineração de Dados Educacionais. 
\linebreak\linebreak
\textbf{TÍTULO:} Modelo de Ambiente Virtual de Aprendizado suportado por CS.


%\setSpacing{1.5} %%set espaçamento entre linhas de 1.5cm
\section{Introdução}

Modalidades de ensino por Educação à Distância (EaD) caracterizam práticas pedagógicas diferenciadas no processo de ensino e aprendizagem, de forma que tal modalidade utiliza-se das tecnologias de informação e comunicação visando facilitar a aquisição do conhecimento \cite{Rabelo_et_al2017}.

Ambientes Virtuais de Aprendizagem (AVA’s) reproduzem modelagens e instruções que possam inferir o estado do aprendizado de cada estudante, \citeonline{Rabelo_et_al2017} reiteram que essas plataformas suportam a interação entre alunos e o ambiente educacional, gerando assim valores expressivos de dados, estes quais quando gerenciados e analisados podem recomendar ampliações sobre os usuários e sua dinâmica de interação com o sistema.

\citeonline{Falci_et_al_2018} descrevem que o relacionamento entre professores e alunos nestas plataformas dá-se pela troca de materiais, discussão em fóruns e chats, todavia, estes meios por muitas vezes não são suficientes o bastante para que o discente possa extrair o máximo de conhecimento dos assuntos trabalhados.

A Mineração de Dados Educacionais (MDE) são a aplicação das técnicas de Mineração de Dados em dados oriundos de ambientes educacionais \cite{Romero_Ventura_2013}. \citeonline{Fernandes2017} reforça que o uso destas técnicas são soluções promissoras para a compreensão dos dados extraídos de AVA's. 

O conhecimento extraído do círculo educacional pode ser melhor utilizado a partir de análise consciente do ambiente educacional, \citeonline[p. 13]{Endsley2012} descrevem o entendimento dos sinais presentes em um ambiente de Consciência Situacional (CS). Estar ciente da situação deve ser um componente natural da organização cognitiva humana, e os benefícios que resultam de um melhor entendimento da situação podem ser percebidos desde a pré-história \cite{Roy_Breton_Rousseau_2007}.

\citeonline{Martins2018} propõe um modelo baseado em CS e MDE para o apoio ao ensino em Ambientes Virtuais de Aprendizagem, seu modelo usa aspectos da CS para otimizar a ação das técnicas de MDE sobre o conjunto de dados e posteriormente auxiliar o usuário na tomada de decisão via regras decisórias. Tomando base esta pesquisa, o foco deste trabalho concentra-se no estudo e expansão do modelo no uso de linguagens lógicas e ontologias para a construção de novas regras decisórias e modelagens mentais assim como na criação de um protótipo de software para aplicabilidade do modelo.


\section{Objetivos}

\subsection{Objetivo Geral}

Esta pesquisa possui como objetivo geral a aplicação da CS em AVA's, investigar o uso de linguagens lógicas  e ontologias no modelo proposto por \citeonline{Martins2018}, validar e construir um software baseado no mesmo para teste e aplicação em mundo real.

\subsection{Objetivos Específicos}

\begin{itemize}
	\item Construção do software baseado no modelo para aplicação em dados reais;
	\item Aplicar diferentes abordagens na construção do módulo seletor do modelo;
	\item Avaliar os resultados da aplicação do modelo identificando como a CS pode entregar uma nova perspectiva para o usuário no momento da tomada de decisão.  
\end{itemize}

\section{Motivação e Caracterização do Problema de Pesquisa}

A sobrecarga de informações em um ambiente educacional inunda docentes e discente de dados e situações que devem ser avaliadas diariamente, pesquisas aprofundam-se na descoberta de conhecimento por meio da MDE, onde o uso de técnicas de processamento de dados e algoritmos aplicados resultam em conhecimento sobre aquela situação específica.

Apesar da MDE prover resultados satisfatórios quando aplicado em AVA's por vezes o uso dos algoritmos restringe-se a situação específica para a qual foi desenvolvida, nota-se então dentro da literatura a vasta abordagem sobre performances e execução dos algoritmos aplicado sobre o ambiente educacional em análises comparativas. 

O uso da Consciência Situacional procura dar uma nova perspectiva sobre o uso da MDE em AVA's, visto o grande uso de CS no auxílio de tomadas de decisão em situações críticas, o seu uso poderá contribuir significativamente nas definições de quais técnicas terão resultados mais expressivos quando aplicados naquela configuração de estados.

O modelo descrito por \citeonline{Martins2018} descreve como a CS e a MDE podem somar forças para potencializar os resultados dentro de um ambiente educacional, entretanto seu trabalho não passou de uma ideia, não sendo testado com dados reais. O autor ainda deixa aberto um leque de possibilidades que podem ser estudadas para utilização com a CS como o uso de ontologias. Deste modo, este trabalho busca resolver as seguintes questões:

\begin{itemize}	
	\item \textbf{Como pode ser expandido o modelo proposto por \citeonline{Martins2018}?;}
	\item \textbf{Este modelo realmente pode trazer benefícios quando aplicados a dados reais?.}	 
\end{itemize}

\section{Justificativa}

A presente proposta de pesquisa justifica-se por perceber o grande crescimento da modalidade de cursos de Educação a Distância (EAD), tal modalidade vem consolidando como importante ferramenta de capacitação independentemente de tempo e localização de seus usuários. Os AVA's podem gerar informações sobre o processo de aprendizagem do estudantes, resultados estes da análise dos dados decorrentes da interação do usuário com a plataforma \cite{Fernandes2017}.

\citeonline{Leite_et_al_2016} reforçam o uso de MDE como soluções promissoras para compreensão de informações nas base de dados em AVA’s, \cite{Rabelo_et_al2017} destacam o seu uso na descoberta de padrões e informações novas sobre conjuntos de dados relacionados aos ambientes de aprendizagem, suas estruturas e personagens.

\citeonline{Falci_et_al_2018} reitera que sistemas com técnicas de Inteligência Artificial (IA) podem facilitar a comunicação entre computador e estudante, permitindo a criação de ambientes com conteúdo adaptativo ao modelo cognitivo do aluno. 

Pesquisas focam na aplicação da CS em ambiente de características dinâmicas, processos estocásticos e eventualmente de resultados incertos, dado o comportamento humano com relevante grau de impacto ao sistema e pelo tempo como fator crítico de alto nível \cite{Berti2017}.

Dessa maneira, construir modelos de AVA's que utilizam tecnologias baseadas em MDE, CS e demais áreas relacionadas  a IA podem facilitar a tomada de decisão, potencializando o processo de aprendizagem, graças a geração de dados mais claros e assertivos que auxiliam o professor no correto entendimento do ambiente.


\section{Desenvolvimento Teórico}

\section{Procedimentos Metodológicos}

\section{Possíveis Contribuições}



% ----------------------------------------------------------
% Referências bibliográficas
% ----------------------------------------------------------
\bibliography{library}

\end{document}